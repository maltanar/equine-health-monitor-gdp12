\chapter{Conclusions}
\section{Success}
A working prototype for an equine health monitor has been built. The project can be further developed into a commercial product. It also provides many research possibilities in the related fields. 

\begin{itemize}
\item it has a potential for further research, lots of interesting possibilities
\item well documented, structured for future developers
\item commercial success potential
\end{itemize}


Objectives specified for the project are met. Scalability was a desired feature to be able to monitor multiple horses simultaneously. A distributed system with one base station and multiple monitoring devices was designed meet this requirement. Power consumption was critical for the battery-powered monitoring devices. Long battery life is achieved through power-aware design with the help of periodic-sampling strategy and minimal processing on the monitoring device. The collected data is accessible via web and this enables users to retrieve data easily.  

\begin{itemize}
\item Long battery life is achieved through the power-aware design
\begin{itemize}
\item Periodic sampling
\item sleep modes
\item minimal processing on the monitoring device
\end{itemize}
\item distributed system addresses the scalability problem
\item web server done for easy accessibility
\item a low-cost, splash proof case for pcb. Stethoscope head from mechanical workshop for the microphone. But a special design can be made, e.g. with straps etc. PCB can be made smaller.
\end{itemize}

The team had to face some problems and managed to solve most of them in a quick and efficient way. The situation of the sixth member was not clear, therefore the team did the task distribution in a way that it would minimize the problems in case of late arrival or quitting. After he quitted, the team discarded or reallocated his tasks. 

Late assignment of the project to the group delayed the development stage however the team managed to meet the deadlines although it affected some of the design decisions... 

The PCB production was not on time, therefore we ordered more PCB’s from other manufacturers when the deadline we set ourselves for PCB production approached. 

Since none of the team members has a background in bioscience or a related field, consultation to an expert was needed and the team used the information gathered from consultation sessions to make a more realistic approach to the problem. 

The team managed its human and hardware resources well by parallelizing tasks. Testing each unit enabled a faster integration. Integration on DK before the PCB arrived made PCB integration and debugging easier. 

The team did a good budget management to reduce costs. We requested sample ICs from the manufacturers/suppliers. This was especially important for ANT development kit which is quite expensive. Also, we joined a design competition and won a starter kit containing MCU we are using.

Things we managed well:
\begin{itemize}
\item sixth member being far and then quitting
\item late start of the project
\item taking quick action against pcb production delay
\item consulting the experts for a more realistic approach  
\item parallelization of tasks, using developers and the devices efficiently
\item testing each unit and integration on DK before PCB arrived made the PCB integration stage faster
\item ordering samples, ANT module was very expensive and we managed to borrow one 
\end{itemize}

\section{Criticism}
The team did not have much time to perform an in depth research to choose the optimum tools and components for the project. During the development stage some lessons learned:
\TODO{}
\begin{itemize}
\item Zigbee was chosen because... the throughput of X can be achieved with zigbee. another wireless protocol maybe better (Bluetooth?)..
\item some components arrived much later than expected. that delayed prototyping subsystems. what could be done for that? backup plans? but we learned our lesson and didn’t do the same mistake when pcb manufacturing was delayed
\item Gantt chart was kind of unrealistic, integration and building subsystems are assumed to be finished at the same time
\end{itemize}


\section{Future Work}
\label{chap:future_work}
The scope of this project is rather broad and the size of the project is quite large given the limited timeframe. For this and other reasons mentioned throughout the report the state of the system is far from being perfect, yet it is a functional prototype of the envisioned Equine Health Monitor. 

We would like to use this chapter to discuss solutions to problems in the system design and implementation that we discovered, and propose a number of improvements that can be made to the system to increase its usability and extend the possible applications.


\subsection{Solutions for design problems}
During the testing phase of the wireless communication module of our system we realized, that the chosen solution is not able to deliver the bandwidth that we require to transfer audio recording from the monitoring devices to the base station in a reasonable amount of time. 

The problem is that the ZigBee protocol is designed for low bandwidth requirements, for even it is a low power protocol, the energy cost per transmitted byte is a lot higher than for other wireless solutions, such as 802.11 or Bluetooth.

Because we do not require a constant connection, but are able transmit data periodically in a burst transmit style, it is feasible to use an alternative solution and run it at a low duty cycle. 

To be more specific in terms of our design, there are pin compatible replacements for XBee devices that implement the 802.11 standard and the manufacturer of the XBee devices is working on a 802.11 based XBee version as well (currently available as a development kit, not retail)

Replacing the XBees with alternative devices doesn’t require modifications to the system design, and because of the Object Oriented design of the software system and well defined interfaces the changes that are required in the software stack are limited to a clearly defined scope.

An audio compression scheme is then recommended if the system is required to perform audio recordings often and/or send this data over the wireless connection. Such a scheme was not implemented in the project mainly the short amount of time available for development and that a deeper knowledge of the signals is required but could be implemented in an improved version of the system.

\subsection{Possible improvements}
The device is designed as an equine health monitor, however the need for a health monitor is not only limited to horses. As there are almost no horse specific parts used in the system, the device could be adapted for the use on human subjects or agents of other mammal species. The only adaptation that has to be made is the usage of a target specific heart rate monitor. 

The scope of the project is rather large. It is possible improve some of the features to evolve it into a more specialized system. One possible solution could be a tracking system using GPS and accelerometer data. These data can also be used for health-related applications such as tracking the movement of the animals and detecting extraordinary situations. 

Another field of application could be professional horse training, to monitor the heart rate in relation to the covered distance over an average of the velocity which allows to analyze the improvements of a horse’s stamina over time.

As the system works as a health profiler, it is possible extend it by adding signal analysis features to perform autonomous diagnosis or issue warnings when there is a problem with the vital signs. Further research has to be undertaken to determine relations of parameters and indicators for certain diseases.

In the current implementation, the gut sound monitor works periodically. The problem with this approach is introduced noise when the horse is moving. One way to improve the behavior is to disabling the audio recording conditionally, if the horse is moving and delay it until the horse stays at a stable position for a period of time. The (already available) accelerometer and GPS data can be used to implement this feature.

In the current state of the system the webinterface is not very user friendly when it comes to analyzing large amounts of collected data. This could be improved by using advances web interface technologies like Ajax to build dynamic version of the website that behaves more like a desktop application than a website.

As accessibility is an important goal of the project it would be a useful addition to extend the website so that a handheld optimized version of the website is delivered when the user accesses the base station with a smartphone.

The limitations of the used XBee devices was discussed in section \ref{sec:wireless_testing}. To be able to transmit audio recordings the ZigBee network should be replaced with a faster wireless connection. Pin compatible XBee replacement ICs based on the 802.11 standard exist and require only a modification on the software part of the system.

Even though the system was designed with energy efficiency in mind from the very beginning, there are subsystems that require further optimization for low energy consumption. The XBee devices are not sent to sleep in the current system, because the delay before they manage to reestablish a connection is too long.