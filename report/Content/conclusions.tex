\chapter{Conclusions}
\section{Success}
A working prototype for an equine health monitor has been built. The project can be further developed into a commercial product. It also provides many research possibilities in the related fields. 

\begin{itemize}
\item it has a potential for further research, lots of interesting possibilities
\item well documented, structured for future developers
\item commercial success potential
\end{itemize}


Objectives specified for the project are met. Scalability was a desired feature to be able to monitor multiple horses simultaneously. A distributed system with one base station and multiple monitoring devices was designed meet this requirement. Power consumption was critical for the battery-powered monitoring devices. Long battery life is achieved through power-aware design with the help of periodic-sampling strategy and minimal processing on the monitoring device. The collected data is accessible via web and this enables users to retrieve data easily.  

\begin{itemize}
\item Long battery life is achieved through the power-aware design
\begin{itemize}
\item Periodic sampling
\item sleep modes
\item minimal processing on the monitoring device
\end{itemize}
\item distributed system addresses the scalability problem
\item web server done for easy accessibility
\item a low-cost, splash proof case for pcb. Stethoscope head from mechanical workshop for the microphone. But a special design can be made, e.g. with straps etc. PCB can be made smaller.
\end{itemize}

The team had to face some problems and managed to solve most of them in a quick and efficient way. The situation of the sixth member was not clear, therefore the team did the task distribution in a way that it would minimize the problems in case of late arrival or quitting. After he quitted, the team discarded or reallocated his tasks. 

Late assignment of the project to the group delayed the development stage however the team managed to meet the deadlines although it affected some of the design decisions... 

The PCB production was not on time, therefore we ordered more PCB’s from other manufacturers when the deadline we set ourselves for PCB production approached. 

Since none of the team members has a background in bioscience or a related field, consultation to an expert was needed and the team used the information gathered from consultation sessions to make a more realistic approach to the problem. 

The team managed its human and hardware resources well by parallelizing tasks. Testing each unit enabled a faster integration. Integration on DK before the PCB arrived made PCB integration and debugging easier. 

The team did a good budget management to reduce costs. We requested sample ICs from the manufacturers/suppliers. This was especially important for ANT development kit which is quite expensive. Also, we joined a design competition and won a starter kit containing MCU we are using.

Things we managed well:
\begin{itemize}
\item sixth member being far and then quitting
\item late start of the project
\item taking quick action against pcb production delay
\item consulting the experts for a more realistic approach  
\item parallelization of tasks, using developers and the devices efficiently
\item testing each unit and integration on DK before PCB arrived made the PCB integration stage faster
\item ordering samples, ANT module was very expensive and we managed to borrow one 
\end{itemize}

\section{Criticism}
The team did not have much time to perform an in depth research to choose the optimum tools and components for the project. During the development stage some lessons learned:
\TODO{}
\begin{itemize}
\item Zigbee was chosen because... the throughput of X can be achieved with zigbee. another wireless protocol maybe better (Bluetooth?)..
\item some components arrived much later than expected. that delayed prototyping subsystems. what could be done for that? backup plans? but we learned our lesson and didn’t do the same mistake when pcb manufacturing was delayed
\item Gantt chart was kind of unrealistic, integration and building subsystems are assumed to be finished at the same time
\end{itemize}
